\documentclass[12pt]{article}

\usepackage{fullpage}
\usepackage{verbatim}
\usepackage{amsmath,amssymb}

\title{Homework \#1}
\author{High Performance Scientific Computing}
\date{AMath 483/583}

\begin{document}

\maketitle

The first homework assignment will be written entirely in Python. It will serve,
in part, as practice for submitting homework as well as a test of your Python,
Numpy, and Matplotlib knowledge and your general programming abilities.

For the portions of the problems marked ``{\bf Report}'' be sure to include your
written responses and any plots requested in a PDF document. Place the document
in the ``{\tt report}'' directory of your repository and give it the name ``{\tt
  report.pdf}''. Please include your name and UWNetID at the top of the
report.

\section*{Exercise \#1}

Consider the function $C : \mathbb{N} \to \mathbb{N}$
\[
  C(n) = 
  \begin{cases}
    n/2,    & \text{if } n \equiv 0 \pmod{2}, \\
    3n + 1, & \text{if } n \equiv 1 \pmod{2}, \\
    1, & \text{if } n = 1.
  \end{cases}
\]
That is, if the input number is even then divide it by two. If the if the input
number is odd then multiply it by three and add one. If the given number is one
then return one.

Given an $n \in \mathbb{N}$ define the sequence,
\[
  n_0 := n, \;\;
  n_1 := C(n_0), \;\;
  n_2 := C(n_1), \;\;
  n_3 := C(n_2), \;\;
  \ldots.
\]
The {\it Collatz conjecture} states that no matter which $n \in \mathbb{N}$ you
begin with the sequence of integers $n_k$ will be finite and end with the number
one. That is,
\[
  S(n) := \{n_0, n_1, \ldots, n_m = 1\}.
\]
This sequence is called a {\it Collatz sequence} and the number $m$ is called
its {\it stopping time}.

For example, the Collatz sequence of $n=6$ is equal to,
\[
  \S(6) = \{6, 3, 10, 5, 16, 8, 4, 2, 1\},
\]
and therefore its stopping time is equal to $m=8$.
\begin{enumerate}
\item Write the body of the function {\tt collatz\_step(n)} defined in the
  module \\* {\tt homework1.exercise1.collatz\_step}. (That is, in the file {\tt
    homework1/exercise1.py} located in the homework repository.) The value of
  this function should be equal to the function $C(n)$ defined above.
\item Write the body of the function {\tt collatz(n)} corresponding to the
  funciton $S(n)$ defined above. This function should output the Collatz
  sequence corresponding to the input $n > 0$.
\item {\bf Report:} For your homework report, create a scatter plot of stopping
  times for all $1 \leq n \leq 5000$. That is, the input integer $n$ should be
  on the horizontal axis and the stopping time should be on the vertical axis.
  {\it Be sure to include appropriate labels and a title. Make the color of the
    data points blue.} Do you think based on this picture that the Collatz
  conjecture is true? Say something about how the picture might be sufficient or
  insufficient to make the conjecture.
\item {\bf Automated Tests:} The hidden test suite will test your
  implementations of {\tt collatz\_step} and {\tt collatz} in the following way:
  \begin{itemize}
  \item Does {\tt collatz\_step} produce the correct output on a variety of
    examples?
  \item Does {\tt collatz\_step} handle the case when $n = 1$ properly?
  \item Does {\tt collatz} produce the correct output on a variety of examples?
  \end{itemize}
\end{enumerate}


\section*{Exercise \#2}

Many scientific computation problems involve finding the root of a given
function. Newton's method is a simple, yet highly effective and popular way to
numerically approximate a ``nearby'' root $x^*$ of a function $f : \mathbb{C}
\to \mathbb{C}$. Define the {\it Newton step function},
\[
  N(f, x_0) =
  \begin{cases}
    x_0 - \frac{f(x_0)}{f'(x_0)}, & \text{if } f'(x_0) \neq 0, \\
    x_0, & \text{if } f'(x_0) = 0,
  \end{cases}
\]
and the {\it Newton iteration function},
\[
  N_k(f, x_0)
  =
  \underbrace{(N \circ N \circ \cdots \circ N)}_{k \text{ times}}(f, x_0).
\]
That is, $N_k(f,x_0)$ is the result of $k$ repeated applications of the Newton
step function to the output of each previous evaluation. In other words, given
an inital guess $x_0$,
\[
  x_1 := N(f,x_0), \;
  x_2 := N(f,x_1), \;
  \ldots, \;
  x_k := N(f,x_{k-1}) = N_k(f,x_0).
\]
The function, $N_k$, satisfies the property,
\[
  \lim_{k \to \infty} N_k(f, x_0) = x^* \quad \text{where} \quad f(x^*) = 0.
\]
That is, the limit $x^*$ is equal to some root of the function $f$.

\begin{enumerate}
\item Write the body of the Python function {\tt newton\_step(f, df, x0)}
  located at the Python module {\tt homework1.exercise1.newton\_step}. (i.e. in
  the file {\tt homework1/exercise1.py} within the homework repo.) To account
  for floating point roundoff error change the condition
  \[
    \text{if } f'(x_0) \neq 0
  \]
  to
  \[
    \text{if } | f'(x_0) | < 10^{-12}
  \]
  in the definition of $N(f,x_0)$.
\item Use your \verb=newton_step= function to write the body of the function
  \verb=newton(f, df, x0)=.
\end{enumerate}

\end{document}